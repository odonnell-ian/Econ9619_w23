\documentclass[12pt]{article}
\usepackage[utf8]{inputenc}
\usepackage{amsmath}
\usepackage{graphicx}
\usepackage{color}
\usepackage{amsfonts}
\usepackage{amssymb}
\usepackage{mathtools}
\usepackage{bbm}
\usepackage[colorlinks=true]{hyperref}
\usepackage[colorinlistoftodos]{todonotes}

\title{Computational Methods\\Problem Set 1}
\author{Ian O'Donnell}
\date{January 23, 2023}
\setlength {\marginparwidth }{2cm} 
\begin{document}

\maketitle

\subsection*{Problem 1}

A competitive equilibrium is a set of prices $ \{p_t , r_t, w_t\}_{t=0}^\infty $, an allocation $ \{k_t^d, n_t^d, y_t\}_{t=0}^\infty $
for the firm, and an allocation $ \{c_t, i_t, x_{t+1}, k_t^s, n_t^s,\}_{t=0}^\infty $ for the household such that: 
\begin{enumerate}
    \item The firm's allocation solves: 
    \begin{align*}
        \max_{k_t, n_t} \pi = \sum_{t=0}^{\infty} p_t [y_t - r_t k_t - w_t n_t] \\
        \text{s.t.} \quad y_t \leq F(k_t, n_t)\\
    \end{align*}

    \item The household's allocation solves: 
    \begin{gather*}
        \max_{c_t, n_t} \sum_{t=0}^{\infty} \beta^t u(c_t) \\
        \text{s.t.} \quad \sum_{t=0}^{\infty} p_t[c_t + i_t] \leq \sum_{t=0}^{\infty} p_t[r_t k_t + w_t n_t] + \pi _t \\
        x_{t+1} = i_t, \quad 0 \leq n_t \leq 1, \quad 0 \leq k_t \leq x_t \\
        k_0 \quad \text{given}, \quad c_t \geq 0, \quad x_{t+1} \geq 0 \\
        \text{TVC:} \quad \lim_{t \rightarrow \infty} p_t k_{t+1} = 0 
    \end{gather*}

    \item All markets clear for all $t$: $k_t^d = k_t^s$, $n_t^d = n_t^s$, $c_t + i_t = y_t$
\end{enumerate}

\subsection*{Problem 2}

The social planner solves: 

\begin{gather*}
    \max_{c_t, k_t, n_t} \sum_{t=0}^{\infty} \beta^t u(c_t) \\
    \text{s.t.} \quad F(k_t, n_t) = c_t + k_{t+1} \\
    \quad c_t \geq 0, \quad k_t \geq 0, \quad 0 \leq n_t \leq 1 \forall t \\
    \quad k_0 \quad \text{given} \\
    \quad \text{TVC:} \quad \lim_{t \rightarrow \infty} \beta^t u'(F(k_t, n_t) - k_{t+1})F_k(k_t, n_t)k_t = 0 
\end{gather*}

\subsection*{Problem 3}

We will show that if the allocation and prices $ \{c_t^e, k_{t+1}^e, p_t^e, w_t^e, r_t^e \}_{t=0}^\infty $ is an 
equilibrium, then $\{c_t^e, k_{t+1}^e \}_{t=0}^\infty $ is a solution to the planners problem. Afterwards, we will also show
that if $\{c_t^*, k_{t+1}^* \}_{t=0}^\infty $ is a solution to the planners problem, then we can also find prices that 
support it as an equilibrium. 

First, suppose that there exists an allocation $\{c_t', k_{t+1}' \}_{t=0}^\infty $ that provides a higher utility level. 
Then it must violate consumers budget constraint or it would have been chosen by the consumer. The implies that the household 
earns profits from the firm. This violates the firm's profit maximizing assumption. This is a contradiction. 

Next, suppose that $\{c_t^*, k_{t+1}^* \}_{t=0}^\infty $ is a solution to the planners problem. First note that the partial
derivatives of the production function give $r_t$ and $w_t$. Then let $f(k_t) = F(k_t, 1)$. We then suppose that 
$p_{t+1} = \frac{p_t}{f'(k_t)}$. This ensures that the first order conditions in the planners problem and the households
problem can be shown to be equivalent. Then the TVCs can also be shown to be equivalent using the relative prices and the 
first order conditions. 

\subsection*{Problem 4}

The planners dynamic programming problem is: 

\begin{equation*}
    v(k) = \max_{0 \leq k' \leq f(k)} u(f(k) - k') + \beta v(k')
\end{equation*}

\subsection*{Problem 5}

To analytically solve the dynamic programming problem, we will use guess and verify with following initial guess:
$v(k) = A + B \ln(k)$. 

\begin{gather*}
    v(k) = \max_{0 \leq k' \leq zk^\alpha} u(zk^\alpha - k') + \beta v(A + B \ln(k)) \\
    \text{FOC:} \frac{-1}{zk^\alpha - k'} + \frac{\beta B}{k'} = 0 \\
    k' = \frac{\beta B z k^\alpha}{1 + \beta B} \\
\end{gather*}

Next, evaluate the the objective function at the optimal value for $k'$

\begin{gather*}
    \ln(zk^\alpha - \frac{\beta B z k^\alpha}{1 + \beta B} ) + \beta(A + B \ln (\frac{\beta B z k^\alpha}{1 + \beta B})) \\
    \implies B = \frac{\alpha}{1 - \beta \alpha} \\
    \implies A = \frac{1}{1 - \beta}[ln(1-\beta \alpha) + \frac{\beta \alpha}{1 - \beta \alpha} \ln(\beta \alpha) + (\frac{\alpha}{1 - \beta \alpha})\ln(z)] \\
\end{gather*}

\begin{equation*}
    \implies g(k) = \frac{\beta \alpha}{1 - \beta \alpha} (1 - \beta \alpha) z k^\alpha = \beta \alpha z k^\alpha
\end{equation*}

\subsection*{Problem 6}

Steady States: 

\begin{equation*}
    \implies \bar{k} = \beta \alpha z \bar{k}^\alpha \implies \bar{k} = (\beta \alpha z)^{\frac{1}{1-\alpha}}
\end{equation*}

\begin{gather*}
    \bar{c} = f(\bar{k}) - \bar{k}, \quad \bar{y} = f(\bar{k}) \\
    \bar{r} = F_k(\bar{k}, 1), \quad \bar{w} = F_n(\bar{k}, 1)
\end{gather*}

\subsection*{Problem 7}

\begin{center}
    \includegraphics*[scale = 0.5]{fig1.png}
\end{center}

\begin{center}
    \includegraphics*[scale = 0.5]{fig2.png}
\end{center}

\begin{center}
    \includegraphics*[scale = 0.5]{fig3.png}
\end{center}

\begin{center}
    \includegraphics*[scale = 0.5]{fig4.png}
\end{center}

\begin{center}
    \includegraphics*[scale = 0.5]{fig5.png}
\end{center}



\end{document}
